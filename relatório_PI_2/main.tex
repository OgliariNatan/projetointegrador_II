\documentclass[a4paper, 12pt]{article}
\usepackage[portuges]{babel}
\usepackage[utf8]{inputenc}
\usepackage{amsmath}
\usepackage{indentfirst}
\usepackage{graphicx}
\usepackage[colorinlistoftodos]{todonotes}
\usepackage{natbib} %pacotes de bibiografia
\usepackage[pdftex]{hyperref} %Para link/url
%\usepackage{pdfpages} % Para adicionar paginas de arquivos .pdf no documento

\usepackage[left=3cm, right=2cm, top=3cm, bottom=2cm]{geometry} %seta as margens do arquivo

\title{Relatorio do projeto Integrador 2} %comentar
\author{Natan Ogliari \thanks{natan.o@aluno.ifsc.edu.br} \and Augusto Danuel Rodrigues \thanks{augusto.dr@aluno.ifsc.edu.br}} %comentar
\date{\today}

\begin{document}
%\maketitle %Adiciona capa automaticamente COMENTAR

\begin{titlepage}
	\begin{center} %Centraliza o texto
		\huge{Instituto Federal de Educação, Ciência e Tecnologia de Santa Catarina}

\vspace{10pt}
\begin{figure}[!ht]
\centering
\includegraphics[height=3cm, width=8cm]{figura/LOGO.jpg} %width=10cm  height=2cm
\end{figure}

        \vspace{85pt}

		\textbf{\LARGE{Projeto integrador II}}\\
		\large{Relatório Final}
		\vspace{160pt}

	\end{center}

	\begin{flushleft}
		\begin{tabbing}
			Alunos:\qquad\qquad\= Augusto Danuel Rodrigues\\
			\>Natan Ogliari\\

			Professor:\> Luiz Azevedo \\
					  \> Fernando Miranda \\

	\end{tabbing}

	\end{flushleft}

	\begin{center}
		\vspace{\fill}
		Florianópolis/SC, 01 de Dezembro de 2017
	\end{center}
\end{titlepage}
%%%%%%%%%%%%%%%%%%%%%%%%%%%%%%%%%%%%%%%%%%%%%%%%%%%%%%%%%%%
\newpage
\tableofcontents
%\listoftables %Lista as tabelas
\thispagestyle{empty}

\newpage
\pagenumbering{arabic}
%%%%%%%%%%%%%%%%%%%%%%%%%%%%%%%%%%%%%%%%%%%%%%%%%%%%%%%%%
\section{Objetivo}
O objetivo deste documento é disponibilizar um modelo padrão de relatório para o IFSC em \LaTeX.
\par Para teste do novo paragrafo.

\section{Equipamentos Utilizados}

Para o experimento\footnote{Um exemplo de rodapé} em questão foram utilizados dois geradores síncronos em que um atuava como motor para configurar um conjunto motor síncrono-gerador síncrono. A função do motor síncrono é a de fornecer potência mecânica ao eixo do gerador para que, então, seja possível realizar os ensaios no gerador síncrono. As placas de dados do motor e do gerador síncrono estão ilustrados nas Figuras\footnote{Um exemplo de rodapé}
\par "par" comando para novo paragrafo.


\section{Procedimento Experimental}

Descrição procedimento


\subsection{subseção1}
tabela exemplo

\begin{table}[htb]
\centering
\begin{tabular}{c|c}
$I_{exc}$ [A] & $V_{a}$ [V] \\ \hline
0.188	& 2.63 		\\\hline
0.88 & 20.55	\\\hline
1.83 & 39.26 \\\hline
2.66 & 55.19 \\\hline
3.78 & 74.94 \\\hline
5.00 & 93.70 \\\hline
6.28 & 109.87 \\\hline
7.55 & 122.28 \\\hline
10.05 & 139.03 \\\hline
11.98 & 148.55 \\\hline
13.30 & 153.92

\end{tabular}
\caption{Pontos coletados para o ensaio à vazio}
\label{PTO:col_vazio}
\end{table}

\section{Inserindo imagens}
Neste topito temos um exemplo de adição de imagens.

\begin{figure}[!h]
\centering
\includegraphics[scale=0.5]{figura/latex_logo.png}
\caption{Logo do LaTeX} \cite{lamport1994latex} %Adiciona uma legenda na imagem
\label{fig:Logo_latex}
\end{figure}

Na Figura \ref{fig:Logo_latex} foi visto como adicionar uma figura, e agora vemos como citar ela no decorrer do documento. Aproveitamos para analisar a citação da Tabela \ref{PTO:col_vazio}.

\section{Análise dos Resultados}

Para inserirmos uma formula utilizamos a seguinte sintase:
\begin{equation}
\int a^{2}=b^{2}+c^{2}
\label{EQ:pitagoras}
\end{equation}
\begin{equation}
\sqrt[4]{N}
\end{equation}
\begin{equation}
\sum_{n=0}^{\infty} a_{n}^{2}
\end{equation}

\begin{quote}
Acima foi visto como inserir uma fórmula, e aqui foi enfatizado a afirmação.\par Observem a numeração automática das fórmulas.
\end{quote}

\begin{equation}
f(t)= \frac{A}{2} + \frac{jA}{2 \pi }
\sum_{\stackrel{-\infty}{n
\neq 0}}^{\infty} \frac{1}{n} \, e^{jn2\pi t}
\end{equation}
\begin{equation}
\int_{-L}^{L} sen \frac{m \pi x}{2}\,sen \frac{n \pi x}{2}\,dx =
\left \{
\begin{array}{cc}
0, & m \neq n \\
1, & m = n \\
\end{array}
\right.
\end{equation}

Um dos fatores importentes neste modelo de documentação é a propria citação das imagens/figuras no próprio texto

\section{Conclusão}

\cite{burns1961management}

conclusão\footnote{Um exemplo de rodapé}

Para exemplo de \href{www.google.com.br}{link} adicionado no texto. Ou podemos deixar o link descrito no texto \url{http://florianopolis.ifsc.edu.br/}
%\href{https://aprendolatex.wordpress.com}{Blog de \LaTeX}

Exemplos de referência Journal articles \cite[cita a pagina 789]{batista2015embarcaccao}, the Digital
Object Identifier (DOI) of the cited literature (which should be added
at the end of the reference in question if available).

%\newpage  %Nova Pagina
%Referencias do documento
\bibliographystyle{abbrvnat} % or try abbrvnat or unsrtnat or plainnat
\bibliography{referencia} % refers to referencia.bib

\end{document}
